\documentclass[11pt]{article}

\usepackage{amsfonts}
\usepackage{amsmath}
\usepackage[total={7in,9in}]{geometry}
\usepackage{amsthm}
\usepackage{graphicx}
\usepackage{lmodern}
\usepackage{hyperref}

\title{Zero knowledge ACTUS for holistic financial data management}
\author{ \\ Morgan Thomas \\ Casper Association \\ morgan@casper.network }


\begin{document}

\maketitle

The ability of financial instutitions to know the risks they are exposed to
depends on their access to high quality data which they can use as input to
models and analyses. To be high quality, the data must be accurate, and also
organized. Accurate but disorganized data is not useful for modeling and analysis
until it is organized, because only well organized data can be used as input to
rigorous, systematic processes which make sense of the data.

Organizing financial data involves collating data from diverse sources.
The problem of collating financial data, which may appear to be a boring problem
at first blush, is not only systematically important but also outrageously
complicated. To get a sense of the scale of the problem, let's consider one
relatively tiny piece of it, which is quite extensive in itself. Let's consider
the problem of presenting the prices of one equity on all of the exchanges
where it is listed, given the price feeds for all of the exchanges. To do this,
we need to be able to map the name or ticker symbol of an equity on one exchange
to the name or ticker symbol of the same equity on any other exchange. The natural
approach is to choose a standard symbology for equities and learn to map all of
the symbologies of each exchange onto the standard symbology. There are dozens
of stock exchanges, and thousands of publicly traded companies, each with one
or more share classes. Furthermore, each exchange regularly lists new stocks, and
the symbology mappings must therefore be updated regularly, ideally in an automated
or mostly automated fashion. In practice, this is not a simple problem.

What would make any collation problem much easier is if the data to be collated
already comes in a homogeneous form. If data providers are to arrange for such a
situation, they require open standards for how financial data is to be structured.
ACTUS, the Algorithmic Contract Types Unified Standard, takes a bite out of this
problem by offering a standard format for financial contracts. ACTUS takes the
view that basic financial instruments like cash and equities, as well as
debt instruments and derivatives, are all contracts. Contracts are not the only
type we need in an ontology of financial data, but they are certainly a very
useful type which captures a lot of the data types that need to be expressed
in an ontology of financial data. ACTUS itself is not a complete ontology of
contracts, but reportedly (TODO: who says this?) it is rare for contracts not
to be expressed within the ontology of ACTUS contracts.

\end{document}
